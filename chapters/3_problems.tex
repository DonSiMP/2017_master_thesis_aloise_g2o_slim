\chapter{Typical Problems}\label{ch:problems}
In this Chapter the reader will became familiar with typical formulations of SLAM problems, with a particular focus on 3D environments. Obviously, there are several other instances of the problem that will be not shown since they are not strictly related to this work.

\section{Pose Graphs}\label{sec:pose_graphs}
Pose Graphs represents the backbone of SLAM. In this problem, the state vector $\state = \left(\state_1, ..., \state_N\right)$ is composed by a 2D or 3D isometry, thus, each node of the graph $\state_i$ belongs to $SE(2)$ or $SE(3)$ - i.e. the \textit{Special Euclidean} group of dimension $2$ or $3$.

The measurements are also 2D or 3D isometries that lie in $SE(2)$ or $SE(3)$. Therefore, an edge $\meas_{ij}$ that connects $\state_i$ with $\state_j$ represents the pose of node $j$ expressed in the reference frame of node $i$ - e.g. $\meas_{ij} =\: \T{i}{j}$. 

This problem is very common in SLAM: suppose that we want to estimate the best trajectory of a robot, given only the \textit{odometry} measurements, in 2D. The odometry retrieves robot's motion from state $\state_i$ to $\state_j$ and encodes it into the transformation matrix $\T{i}{j}$. Moreover, supposing that the robot is also able to retrieve \textit{loop-closures}, this would mean that there is an edge between the nodes $i$ and $k$. The related transformation is encoded into the quantity $\meas_{ik} =\: \T{i}{k}$.

Clearly, both states and measurements are non-euclidean. The extended parametrization of those quantities - as it has been state before - is given by an isometry, while a possible \textit{minimal representation} can be a $3$ vector $(t_{x}\: t_{y}\: \theta)^T$. The next step is to define the operators \textit{box-plus} and \textit{box-minus}. Therefore, we introduce the operators $t2v$ and $v2t$ that allow to map an isometry into a $3$ vector and vice versa. Given those operators, we will have the following relations:

\begin{align}
    \label{eq:boxplus_se2}
    \SState \boxplus \dx &= \SState \cdot \v2t(\dx)\\
    \label{eq:boxminus_se2}
    \SState_a \boxminus \SState_b &= \text{t2v}\left(\SState_b^{-1} \, \SState_a\right)
\end{align}

Since the measurement $\meas_{ij}$ expresses pose $\SState_j$ with respect to the reference system of $\SState_i$, the predicted measurement $\meas_{ij}$ can be computed as 

\begin{equation}
    \label{eq:prediction_se2}
    \Pred_{ij} = h_{ij}(\SState) = \SState_i^{-1} \, \SState_j
\end{equation}

\noindent Given the relations \ref{eq:boxminus_se2} and \ref{eq:prediction_se2}, the error is computed as follows:

\begin{align}
    \error_{ij}\left(\SState\right) &= h_{ij}\left(\SState\right) \boxminus \Meas_{ij} = \nonumber \\
    &= \text{t2v}\left(\Meas_{ij}^{-1}\, \SState_i^{-1} \,\SState_j\right)
    \label{eq:error_se2}
\end{align}

Finally, Jacobians must be computed and, to do so, we apply a perturbation to the error function \ref{eq:error_se2}, that leads to the following relation:

\begin{align}
    \error_{ij}\left(\SState \boxplus \dx\right) &= \text{t2v}\left(\Meas_{ij}^{-1} \, \left(\SState_i\cdot\v2t(\dx_i)\right)^{-1}\: \left(\SState_j\cdot\v2t(\dx_j)\right) \right) = \nonumber\\
    &= \text{t2v}\left(\Meas_{ij}^{-1} \, \v2t(\dx_i)^{-1}\: \underbracket{\SState_i^{-1}\SState_j}_{\Pred_{ij}} \:\v2t(\dx_j) \right) = \nonumber \\
    &= \text{t2v}\left(\Meas_{ij}^{-1} \, \v2t(\dx_i)^{-1}\: \Pred_{ij} \:\v2t(\dx_j) \right)
    \label{eq:error_perturbation_se2}
\end{align}

\noindent The remaining part is just the computation of the following partial derivatives:

\begin{align}
    \label{eq:jac_i_se2}
    \jacob_i = \frac{\partial\: \text{t2v}\left(\Meas_{ij}^{-1} \, \v2t(\dx_i)^{-1}\: \Pred_{ij} \:\v2t(\dx_j) \right)}{\partial \dx_i} \Bigg\rvert_{\dx_i = 0, \dx_j=0} \\
    \label{eq:jac_j_se2}
    \jacob_j = \frac{\partial\: \text{t2v}\left(\Meas_{ij}^{-1} \, \v2t(\dx_i)^{-1}\: \Pred_{ij} \:\v2t(\dx_j) \right)}{\partial \dx_j} \Bigg\rvert_{\dx_i = 0, \dx_j=0}
\end{align}

\noindent The final Jacobian will be non-zero only in the blocks relative to variables $\SState_i$ and $\SState_j\,$, in formul\ae:

\begin{equation*}
    \jacob = \left[\zero \: \cdots \: \zero \:\, \jacob_i \,\: \zero \: \cdots \: \zero \:\, \jacob_j \,\: \zero \: \cdot \: \zero\right]
\end{equation*}

The reader might notice that the equation \ref{eq:error_perturbation_se2} is highly non-linear - as also the operators $\text{t2v}$ and $\v2t$ themselves - and this will translate into a less accurate approximation obtained from the first-order Taylor expansion of the error \ref{eq:error_se2} and into the computation complex derivatives given by \ref{eq:jac_i_se2} and \ref{eq:jac_j_se2}. The situation becomes even worse in a 3D environment, which will be better analyzed in the next Chapters. 

\section{Pose-Landmarks Graphs}\label{sec:pose_land_graph}
In this Section we will propose another common formulation of the problem. Landmarks represent the position of salient world points and here are used to optimize the trajectory of the robot, assuming that the position of those //TODO
