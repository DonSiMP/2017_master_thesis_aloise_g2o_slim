\chapter{Introduction}\label{ch:intro}
Mobile robots, to accomplish tasks in real world more easily and in an efficient way, need to have a map of the environment and to localize themselves into the map. Furthermore, in some environment it is not possible to rely on external reference systems - e.g. GPS - and, thus, they can count only on the on-board sensors. \textit{Simultaneous Localization and Mapping} (SLAM) addresses the problem of \textbf{learning the map under pose uncertainty}.

There are many scenarios in which SLAM is fundamental for the accomplishment of a task, not only in pure Robotics. SLAM, in fact, is a common problem in different domains of application. For example, in Robotics it is fundamental for indoor navigation of mobile robots - e.g. an autonomous vacuum cleaner or a service robot in a museum - or to navigate through extreme environments - e.g. underwater rescues or space exploration. Additionally, new technologies that involve different kind of agents - i.e. not robots - are now using SLAM. One of the most trending one is \textit{Augmented Reality} (AR). Always more powerful mobile devices - like smartphones or tablets - are now able to exploit SLAM to deliver stunning virtual experience. Without any doubts, this technology is going to gain always more popularity and to impact on the research in this topic.

As the reader might notice, SLAM is a popular problem and the research community is focusing on in since many years. Several formulations have been proposed through the years and now current state-of-the-art SLAM systems are able to deliver impressive results in real-world scenarios. The most used formulation for SLAM systems is the so called \textit{graph-based SLAM}. In this approach, two sub-systems cooperate with each other to retrieve the best robot trajectory and world configuration given the on-board sensors' measurements. The two sub-systems are:

\begin{enumerate}
    \item \textit{Front-end}: it exploits sensor data to build an hyper-graph whose nodes are either robot poses or the position of salient point in the world;
    \item \textit{Back-end}: it is in charge of performing non-linear optimization of the graph to retrieve the most likely configuration that suits the measurements.
\end{enumerate}

In this work, we propose a back-end system built from scratch that is able to perform fast and accurate graph optimization for 3D environments. The work is focused on simplicity and minimalism also in its implementation, in order to be comprehensible for non-expert people that what to understand how the system works. Despite its minimalistic fashion, system's results are comparable to the ones of other state-of-the-art systems, thanks to the use of some novel theoretic ideas and to a well-designed implementation. In particular, this work shows the effectiveness of a \textbf{new error function} for $SE(3)$ objects (Section \ref{sec:se3_objects}) and an implementation with \textbf{zero memory copy} during the optimization process (Section \ref{sec:bottlenecks}).

\vspace{20px}

\noindent The remaining of this document is organized as follows:

\begin{itemize}
    \item \textit{Chapter 2}: overview of the problem and of methodologies employed through the years, with a particular focus on noteworthy systems;
    \item \textit{Chapter 3}: problem statement and fundamental theoretic concept related to the non-linear optimization problem;
    \item \textit{Chapter 4}: sketch of the most common SLAM problem formulations;
    \item \textit{Chapter 5}: deeper examination of 3D problem formulations and further analysis of the proposed approach;
    \item \textit{Chapter 6}: details about code design and implementation choice;
    \item \textit{Chapter 7}: focus on two full SLAM systems that uses the proposed system as on-line back-end
    \item \textit{Chapter 8}: final considerations and possible future investigations. 
\end{itemize}