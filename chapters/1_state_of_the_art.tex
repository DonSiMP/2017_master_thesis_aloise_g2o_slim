\chapter{State of the Art}\label{ch:related}
\lettrine[lines=2]{S}{}imultaneous Localization and Mapping represents a well known complex mathematical problem, based on non-linear optimization. It has been studied by the robotics community since the 80s \cite{durrant2006simultaneous} \cite{bailey2006simultaneous}; during this early stage, its statistical formulation has been investigated, proposing interesting results that will constitute the baseline basically for all the future SLAM systems.

After some years, in the 90s, early solutions to the SLAM problem start to arise. The first systems able to produce appreciable results in terms of speed and accuracy were based on \textit{Extended Kalman Filters} (EKF) \cite{leonard1990dynamic} \cite{dissanayake2001solution}. EKFs allow to deal with problem's non-linearity through effective approximations and to represent multivariate distributions with a small number of parameters. This success encouraged the research community to perform deeper investigations in \textit{filtering} approaches \cite{aulinas2008filtering_review}. \textit{Particle filters} started to gain popularity, in particular \textit{Rao-Blackwellized Particle Filters} \cite{grisetti2005improving}: the work of Montemerlo \textit{et al.} \cite{montemerlo2002fastslam} was the first SLAM system able to deal with thousand of landmarks with a good accuracy. 

However, \textit{filtering} approaches revealed to be not the best answer to the SLAM problem due to the computational complexity of the solution, especially when dimensions grow. Moreover, system's accuracy is affected by the non-linearities, leading to non-optimal solutions. \textit{Maximum A Posteriori} (MAP) starts to be taken in consideration and the community has taken a step back to the work of Lu \textit{et al.} \cite{lu1997globally}.