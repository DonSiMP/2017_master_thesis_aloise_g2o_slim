\chapter{Use Cases}\label{ch:cases}
As already mentioned, the system developed aims to deliver real time performances also when used in real world applications. Furthermore, the system provides some easy-to-embed APIs in order to be easily integrated in a full SLAM pipeline. Therefore, in this Chapter we provide two front-end systems developed in the Lab-Rococo at Sapienza University that actually use this work as back-end. In both cases, our system is used to perform 3D \textit{pose-graph} optimization.

\section{ProSLAM}\label{sec:proslam}
The work of Schlegel \textit{et al.} \cite{schlegel2017proslam} presents a stereo visual system capable of mapping dynamic large-scale environments called ProSLAM. The system is designed with simplicity and modularity in mind and, thus, it is easy to implement and to understand also from people who are not Computer Vision or SLAM experts.

This works is also almost entirely self-contained and employs only a minimal set of external libraries - among which there is the library described in this work.

ProSLAM, despite its simplicity, is able to provide results comparable with other more complex state-of-the-art front-ends. Its goal is to generate a 3D map from the processing of a sequence of stereo-images. The map is intended as a collection of \textit{landmarks} - salient 3D points in the world characterized in its appearances by a unique descriptor - together with the camera trajectory. 

Landmarks acquired in a nearby region define a \textit{local-map}, and each of them constitutes a node of the graph - i.e. a $SE(3)$ transformation matrix. Edges between local maps encode the spatial constraints correlating local maps close in space. Those constraints are generated by two kind of events:

\begin{enumerate}
    \item \textbf{Tracking} of the camera motion between temporal subsequent maps
    \item \textbf{Alignment} of local maps acquired at distant times as a consequence of \textit{re-localization} events - i.e. \textit{loop-closures}.
\end{enumerate}

Re-localization is more complex to address with respect to tracking. In fact, to achieve this goal it is necessary to compare descriptors of all the local-maps. This is an expensive operation, and it is efficiently performed by the \textit{Hamming Binary Search Tree} (HBST) \cite{schlegel2016hbst} library. The system periodically triggers graph optimization and this helps also the re-localization process.

It is good to notice that ProSLAM runs on \textit{single thread}, delivering performances comparable with other more complex and multi-thread systems - e.g. \cite{mur2017orb-slam2}.

TODO results

\section{ProSLAM-stud}\label{sec:froslam}
Colosi \textit{et al.} propose in their work ProSLAM-stud \cite{colosi2017froslam} a further iteration on minimalism from original ProSLAM. It is a \textit{Master thesis} work, therefore its approach is more didactic than the original one. However, the system still delivers quite good performances both in terms of speed and accuracy. 

This systems embeds a new \textit{tracking} method based on KD-Tree, that improves the \textit{open-loop estimation} and is able to provide a boost in speed with respect to ProSLAM. 

Having a good open-loop estimate translates into a better initial guess of the optimization problem and, therefore, the whole system can benefit of this novelty. ProSLAM-stud keeps the single-treaded implementation and despite the minimal approach, it can push up to more than 80Hz - on average.

TODO results